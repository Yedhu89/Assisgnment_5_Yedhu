% Options for packages loaded elsewhere
\PassOptionsToPackage{unicode}{hyperref}
\PassOptionsToPackage{hyphens}{url}
%
\documentclass[
  12pt,
]{article}
\usepackage{amsmath,amssymb}
\usepackage{lmodern}
\usepackage{ifxetex,ifluatex}
\ifnum 0\ifxetex 1\fi\ifluatex 1\fi=0 % if pdftex
  \usepackage[T1]{fontenc}
  \usepackage[utf8]{inputenc}
  \usepackage{textcomp} % provide euro and other symbols
\else % if luatex or xetex
  \usepackage{unicode-math}
  \defaultfontfeatures{Scale=MatchLowercase}
  \defaultfontfeatures[\rmfamily]{Ligatures=TeX,Scale=1}
\fi
% Use upquote if available, for straight quotes in verbatim environments
\IfFileExists{upquote.sty}{\usepackage{upquote}}{}
\IfFileExists{microtype.sty}{% use microtype if available
  \usepackage[]{microtype}
  \UseMicrotypeSet[protrusion]{basicmath} % disable protrusion for tt fonts
}{}
\makeatletter
\@ifundefined{KOMAClassName}{% if non-KOMA class
  \IfFileExists{parskip.sty}{%
    \usepackage{parskip}
  }{% else
    \setlength{\parindent}{0pt}
    \setlength{\parskip}{6pt plus 2pt minus 1pt}}
}{% if KOMA class
  \KOMAoptions{parskip=half}}
\makeatother
\usepackage{xcolor}
\IfFileExists{xurl.sty}{\usepackage{xurl}}{} % add URL line breaks if available
\IfFileExists{bookmark.sty}{\usepackage{bookmark}}{\usepackage{hyperref}}
\hypersetup{
  pdftitle={Assignment 5},
  pdfauthor={Yedhu Sanil Kumar},
  hidelinks,
  pdfcreator={LaTeX via pandoc}}
\urlstyle{same} % disable monospaced font for URLs
\usepackage[margin=1in]{geometry}
\usepackage{graphicx}
\makeatletter
\def\maxwidth{\ifdim\Gin@nat@width>\linewidth\linewidth\else\Gin@nat@width\fi}
\def\maxheight{\ifdim\Gin@nat@height>\textheight\textheight\else\Gin@nat@height\fi}
\makeatother
% Scale images if necessary, so that they will not overflow the page
% margins by default, and it is still possible to overwrite the defaults
% using explicit options in \includegraphics[width, height, ...]{}
\setkeys{Gin}{width=\maxwidth,height=\maxheight,keepaspectratio}
% Set default figure placement to htbp
\makeatletter
\def\fps@figure{htbp}
\makeatother
\setlength{\emergencystretch}{3em} % prevent overfull lines
\providecommand{\tightlist}{%
  \setlength{\itemsep}{0pt}\setlength{\parskip}{0pt}}
\setcounter{secnumdepth}{-\maxdimen} % remove section numbering
\ifluatex
  \usepackage{selnolig}  % disable illegal ligatures
\fi

\title{Assignment 5}
\author{Yedhu Sanil Kumar}
\date{10/16/2021}

\begin{document}
\maketitle

\hypertarget{enhancing-the-productivity-of-hybrid-poplar-populus-hybrid-and-switchgrass-panicum-virgatum-l.-by-the-application-of-beneficial-soil-microbes-and-a-seaweed-extract.}{%
\subsubsection{Enhancing the productivity of hybrid Poplar (Populus×
hybrid) and Switchgrass (Panicum virgatum L.) by the application of
beneficial soil microbes and a seaweed
extract.}\label{enhancing-the-productivity-of-hybrid-poplar-populus-hybrid-and-switchgrass-panicum-virgatum-l.-by-the-application-of-beneficial-soil-microbes-and-a-seaweed-extract.}}

Fei, H., Crouse, M., Papadopoulos, Y., \& Vessey, J. K. (2017).
Enhancing the productivity of hybrid Poplar (Populus× hybrid) and
Switchgrass (Panicum virgatum L.) by the application of beneficial soil
microbes and a seaweed extract. Biomass and bioenergy, 107, 122-134.

\url{https://github.com/Yedhu89/Assisgnment_5_Yedhu.git}

\hypertarget{introduction}{%
\paragraph{Introduction:}\label{introduction}}

The study by Fei et al.,(2017) carries out an investigation on the
effects of three beneficial soil microbes Azospirillum brasilense,
Pencillium bilaii, Variovorax paradoxus, and Ascophyllum nodosum
(Seaweed) tested on the growth of three clones of hybrid poplars and two
cultivars of Switchgrass grown under greenhouse conditions and on
marginal land. Good quality and high yielding feedstocks on marginal
lands with low inputs could help lower costs.Beneficial soil microbes
have many advantages to plants like enhanced growth, at low costs
(Vessey, J. K. 2003). Inoculating biofuel feedstock crop species with
such microbes can increase the productivity and lower the costs of
biofuel feedstocks.This experiment was done to enhance to yield and
productivity of the biomass crops by the application of beneficial soil
microbes and a seaweed. The application of beneficial soil microbes and
plant supplements enhances productivity of biomass feedstocks and help
lower cost of biofuel.

\hypertarget{methods}{%
\paragraph{Methods:}\label{methods}}

Three poplar clones and two cultivars of switchgrass were selcted for
the experiment. The hybrid poplar clones were Hill, Okanese and Walker
from the Agriculture and Agri-Food Canada (AAFC) Saskatchewan.The
switchgrass cultivars were a wild type and the cultivar `Cave-In-Rock'.
The plants were grown in both greenhouse and field to evaluate the
treatment effects and were treated with soil microbes and seaweed
extract (A.nodosum). After 1 month of growth, each plant was inoculated
with 2 ml of bacterial broth (108 cfu ml−1), and 1 ml of ANE by soil
drenching. After 3 months of growth, poplar stems, leaves and roots, and
switchgrass shoots and roots were harvested separately, and each plant
sample was ground into a fine powder for nutrient analyses.
Concentrations of Nitrogen (N), Phosphorus (P) and Potassium (K) in the
tissues were determined by Laboratory Service, Department of
Agriculture, Nova Scotia.

\hypertarget{results}{%
\paragraph{Results:}\label{results}}

The plants were harvested after 3 months of growth from both setting.
The results show that Biomass in poplar cultivars like Okanese and
Walker under both conditions was high in treatments with A.brasilense
and P. bilaii by 23\% and 26\%. There were no treatment effects in clone
`Hill', thereby indicating a genotype-specific effect. For switchgrass,
Biomass in `Cave-In-Rock' under greenhouse inoculated with A. brasilense
or P. bilaii was higher by 18\% compared to other treatments, but there
were no significant increases in the field trials. No treatments effects
on biomass in the Wildtype, again indicating a genotype-specific effect.

\hypertarget{conclusion}{%
\paragraph{Conclusion:}\label{conclusion}}

The finals results concluded that under both the greenhouse and field
growth conditions, suggest that in some cases A. brasilense N8 and P.
bilaii can enhance growth and/or nutrient accumulation of hybrid poplar
clone `Okanese' and clone `Walker' and switchgrass cv. `Cave-In-Rock'.
Although less conclusive, there is also some evidence of positive
effects of the A. nodosum extract on the growth and nutrient content in
poplar. The results from the both the greenhouse and field shows A.
brasilense N8 and P. bilaii can increase growth and nutrient
accumulation for poplar clones `Okanese' and `Walker' and for
switchgrass `Cave-In-Rock'. This approach may enable a more sustainable
production of the biomass feedstock crops on marginal soils where no
nutrient supply and periodic drought are present. This also helps to
reduce the cost for Biofuel production.

\hypertarget{graphs}{%
\paragraph{Graphs}\label{graphs}}

\includegraphics{Assignment_5_Yedhu_files/figure-latex/pressure-1.pdf}
\includegraphics{Assignment_5_Yedhu_files/figure-latex/pressure-2.pdf}

\hypertarget{references}{%
\paragraph{References:}\label{references}}

Vessey, J. K. (2003). Plant growth promoting rhizobacteria as
biofertilizers. Plant and soil, 255(2), 571-586.

Fei, H., Crouse, M., Papadopoulos, Y., \& Vessey, J. K. (2017).
Enhancing the productivity of hybrid poplar (Populus× hybrid) and
switchgrass (Panicum virgatum L.) by the application of beneficial soil
microbes and a seaweed extract. Biomass and bioenergy, 107, 122-134.

Hoagland, D. R., \& Arnon, D. I. (1950). The water-culture method for
growing plants without soil. Circular. California agricultural
experiment station, 347(2nd edit).

\end{document}
